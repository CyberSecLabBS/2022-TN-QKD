\chapter*{Conclusione}
In questa relazione ho analizzato la tecnologia della Quantum Key Distribution, descrivendo i protocolli più famosi e le loro caratteristiche. Sono stati evidenziati i punti di forza e debolezza di questa nuova tecnologia, in modo tale da avere una visione più ampia dell'argomento. Abbiamo visto quindi che uno dei più grandi benefici della QKD `e quello di generare e distribuire chiavi sicure attraverso canali non sicuri. Ci sono però ancora delle limitazioni come il costo della tecnologia o il fatto che la QKD richiede un hardware dedicato.. È stata posta particolare attenzione a due protocolli crittografici che permettono lo scambio di chiavi, DH e RSA, grazie ai quali è stato possibile effettuare confronti per quanto riguarda la sicurezza di QKD. È stato possibile osservare che sia DH che RSA possiedono vulnerabilità sfruttabili con opportune risorse quantistiche e che l'utilizzo di QKD potrebbe mitigare alcuni possibili attacchi. Infine, ho mostrato un esempio di come questa tecnologia possa essere implementata nel protocollo TLS (nello specifico nell'Handshake TLS), in modo tale da aumentare la sicurezza e la segretezza delle comunicazioni. È necessario sottolineare però che l'utilizzo di protocolli di QKD comporta diverse problematiche, le quali non possono essere trascurate. La QKD è una tecnologia ancora in fase sperimentale e sicuramente grazie agli sviluppi futuri sarà possibile determinare l'effettiva validità ed efficacia di questa soluzione