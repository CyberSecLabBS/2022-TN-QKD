\setcounter{page}{1}
\chapter*{Introduzione}
Negli ultimi tre decenni, la crittografia è diventata una componente indispensabile per la comunicazione tra individui distanti geograficamente. In un mondo così connesso, la capacità di individui, aziende e governi di comunicare in modo sicuro è fondamentale. 
Molti dei protocolli di comunicazione più importanti si basano principalmente su alcune funzionalità crittografiche tra cui: \textbf{crittografia a chiave pubblica}, \textbf{firme digitali} (funzioni hash) e \textbf{scambio di chiavi} (in questo documento tratteremo quest'ultimo). La crittografia a chiave pubblica attualmente in uso si basa su problemi che coinvolgono la fattorizzazione in numeri primi (RSA), i logaritmi discreti (Diffie-Hellman) e le curve ellittiche (ECC).
Anche se questi sembrano problemi diversi, in realtà sono tutti casi di un problema generale legato alla difficoltà di fattorizzazione in numeri primi.\\
Questo problema è difficile da risolvere, specialmente con algoritmi classici che hanno una complessità cosiddetta (sub)esponenziale. Ci vorrebbero anni per rompere l’attuale crittografia a chiave pubblica anche con il più potente dei computer, supponendo che il sistema sia implementato correttamente.\\
Nel 1994, Peter Shor ha dimostrato che i computer quantistici, una nuova tecnologia che sfrutta le proprietà fisiche della materia e dell'energia per eseguire calcoli, possono risolvere efficacemente ciascuno di questi problemi, rendendo così impotenti tutti i crittosistemi a chiave pubblica basati su tali presupposti \cite{shor_polynomial-time_1997}. In particolare, è stato dimostrato che l'algoritmo di Shor, se implementato su un calcolatore quantistico che utilizza un numero sufficiente di qbit, è in grado di risolvere il problema del logaritmo discreto e della fattorizzazione in numeri primi in tempo \textbf{polinomiale}, con tempi che crescono di poco al crescere della lunghezza delle chiavi crittografiche. Se mai verranno costruiti computer quantistici su larga scala, saranno in grado di violare molti dei crittosistemi a chiave pubblica attualmente in uso. Ciò comprometterebbe gravemente la riservatezza e l'integrità delle comunicazioni digitali su Internet e altrove. L'obiettivo della crittografia post-quantistica (\textit{post-quantum cryptography, quantum-resistant cryptography}) è sviluppare sistemi crittografici che siano sicuri sia contro i computer quantistici che classici e possano interagire con i protocolli e le reti di comunicazione esistenti. A questo proposito sono state ideate soluzioni che utilizzano elementi quantistici per arricchire e potenziare i protocolli già esistenti e l'esempio più noto è la distribuzione di chiavi quantistiche (QKD).